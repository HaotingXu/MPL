\documentclass[11pt,a4paper]{ctexart}
%\usepackage{CJKutf8}
%\setCJKmainfont{AR PL UKai CN}
%\setCJKmainfont{AR PL UKai}
\usepackage{geometry}
\usepackage{caption}
\usepackage{graphicx, subfig,float}
\geometry{a4paper,left=4cm,right=4cm}
\usepackage{appendix}
\usepackage{amsmath}
\usepackage{enumerate}
\usepackage{amssymb,color}
\usepackage[colorlinks,linkcolor=red,anchorcolor=blue,citecolor=blue]{hyperref}
\usepackage{slashed}
\usepackage{simplewick}
\usepackage{tikz}
\usepackage{tcolorbox}
\usepackage[T1]{fontenc}
%colors
\def\blacktext#1{{\color{black}#1}}
\def\bluetext#1{{\color{blue}#1}}
\def\redtext#1{{\color{red}#1}}
\def\darkbluetext#1{{\color[rgb]{0,0.2,0.6}#1}}
\def\skybluetext#1{{\color[rgb]{0.2,0.7,1.}#1}}
\def\cyantext#1{{\color[rgb]{0.,0.5,0.5}#1}}
\def\greentext#1{{\color[rgb]{0,0.7,0.1}#1}}
\def\darkgray{\color[rgb]{0.2,0.2,0.2}}
\def\lightgray{\color[rgb]{0.6,0.6,0.6}}
\def\gray{\color[rgb]{0.4,0.4,0.4}}
\def\blue{\color{blue}}
\def\red{\color{red}}
\def\green{\color{green}}
\def\darkgreen{\color[rgb]{0,0.4,0.1}}
\def\darkblue{\color[rgb]{0,0.2,0.6}}
\def\skyblue{\color[rgb]{0.2,0.7,1.}}
%%control
\def\be{\begin{equation}}
\def\ee{\nonumber\end{equation}}
\def\beq{\begin{equation}}
\def\eeq{\end{equation}}
\def\bea{\begin{eqnarray}}
\def\eea{\end{eqnarray}}
\def\bmat#1{\left(\begin{array}{#1}}
\def\emat{\end{array}\right)}
\def\bcase#1{\left\{\begin{array}{#1}}
\def\ecase{\end{array}\right.}
\def\bmini#1{\begin{minipage}{#1\textwidth}}
\def\emini{\end{minipage}}
\def\tbox#1{\begin{tcolorbox}#1\end{tcolorbox}}
\def\pfrac#1#2#3{\left(\frac{\partial #1}{\partial #2}\right)_{#3}}
%%symbols
\def\bropt{\,(\ \ \ )}
\def\sone{$\star$}
\def\stwo{$\star\star$}
\def\sthree{$\star\star\star$}
\def\sfour{$\star\star\star\star$}
\def\sfive{$\star\star\star\star\star$}
\def\rint{{\int_\leftrightarrow}}
\def\roint{{\oint_\leftrightarrow}}
\def\stdHf{{\textit{\r H}_f}}
\def\deltaH{{\Delta \textit{\r H}}}
\def\ii{{\dot{\imath}}}
\def\skipline{{\vskip0.1in}}
\def\skiplines{{\vskip0.2in}}
\def\lagr{{\mathcal{L}}}
\def\hamil{{\mathcal{H}}}
\def\vecv{{\mathbf{v}}}
\def\vecx{{\mathbf{x}}}
\def\vecy{{\mathbf{y}}}
\def\veck{{\mathbf{k}}}
\def\vecp{{\mathbf{p}}}
\def\vecn{{\mathbf{n}}}
\def\vecA{{\mathbf{A}}}
\def\vecP{{\mathbf{P}}}
\def\vecsigma{{\mathbf{\sigma}}}
\def\hatJn{{\hat{J_\vecn}}}
\def\hatJx{{\hat{J_x}}}
\def\hatJy{{\hat{J_y}}}
\def\hatJz{{\hat{J_z}}}
\def\hatj#1{\hat{J_{#1}}}
\def\hatphi{{\hat{\phi}}}
\def\hatq{{\hat{q}}}
\def\hatpi{{\hat{\pi}}}
\def\vel{\upsilon}
\def\Dint{{\mathcal{D}}}
\def\adag{{\hat{a}^\dagger}}
\def\bdag{{\hat{b}^\dagger}}
\def\cdag{{\hat{c}^\dagger}}
\def\ddag{{\hat{d}^\dagger}}
\def\hata{{\hat{a}}}
\def\hatb{{\hat{b}}}
\def\hatc{{\hat{c}}}
\def\hatd{{\hat{d}}}
\def\hatN{{\hat{N}}}
\def\hatH{{\hat{H}}}
\def\hatp{{\hat{p}}}
\def\Fup{{F^{\mu\nu}}}
\def\Fdown{{F_{\mu\nu}}}
\def\newl{\nonumber \\}
\def\vece{\mathrm{e}}
\def\calM{{\mathcal{M}}}
\def\calT{{\mathcal{T}}}
\def\calR{{\mathcal{R}}}
\def\barpsi{\bar{\psi}}
\def\baru{\bar{u}}
\def\barv{\bar{\upsilon}}
\def\qeq{\stackrel{?}{=}}
\def\torder#1{\mathcal{T}\left(#1\right)}
\def\rorder#1{\mathcal{R}\left(#1\right)}
\def\contr#1#2{\contraction{}{#1}{}{#2}#1#2}
\def\trof#1{\mathrm{Tr}\left(#1\right)}
\def\trace{\mathrm{Tr}}
\def\comm#1{\ \ \ \left(\mathrm{used}\ #1\right)}
\def\tcomm#1{\ \ \ (\text{#1})}
\def\slp{\slashed{p}}
\def\slk{\slashed{k}}
\def\calp{{\mathfrak{p}}}
\def\veccalp{\mathbf{\mathfrak{p}}}
\def\Tthree{T_{\tiny \textcircled{3}}}
\def\pthree{p_{\tiny \textcircled{3}}}
\def\dbar{{\,\mathchar'26\mkern-12mu d}}
\def\erf{\mathrm{erf}}
\def\const{\mathrm{constant}}
\def\pheat{\pfrac p{\ln T}V}
\def\vheat{\pfrac V{\ln T}p}
%%units
\def\fdeg{{^\circ \mathrm{F}}}
\def\cdeg{^\circ \mathrm{C}}
\def\atm{\,\mathrm{atm}}
\def\angstrom{\,\text{\AA}}
\def\SIL{\,\mathrm{L}}
\def\SIT{\, \mathrm{T}}
\def\SIkm{\,\mathrm{km}}
\def\SIyr{\,\mathrm{yr}}
\def\SIGyr{\,\mathrm{Gyr}}
\def\SIV{\,\mathrm{V}}
\def\SImV{\,\mathrm{mV}}
\def\SIeV{\,\mathrm{eV}}
\def\SIkeV{\,\mathrm{keV}}
\def\SIMeV{\,\mathrm{MeV}}
\def\SIGeV{\,\mathrm{GeV}}
\def\SIcal{\,\mathrm{cal}}
\def\SIkcal{\,\mathrm{kcal}}
\def\SImol{\,\mathrm{mol}}
\def\SIN{\,\mathrm{N}}
\def\SIHz{\,\mathrm{Hz}}
\def\SIm{\,\mathrm{m}}
\def\SIcm{\,\mathrm{cm}}
\def\SIfm{\,\mathrm{fm}}
\def\SImm{\,\mathrm{mm}}
\def\SInm{\,\mathrm{nm}}
\def\SImum{\,\mathrm{\mu m}}
\def\SIJ{\,\mathrm{J}}
\def\SIW{\,\mathrm{W}}
\def\SIkJ{\,\mathrm{kJ}}
\def\SIs{\,\mathrm{s}}
\def\SImus{\,\mathrm{\mu s}}
\def\SIkg{\,\mathrm{kg}}
\def\SIg{\,\mathrm{g}}
\def\SIK{\,\mathrm{K}}
\def\SImmHg{\,\mathrm{mmHg}}
\def\SIPa{\,\mathrm{Pa}}
\def\rad{\,\mathrm{rad}}

\graphicspath{{figure/}}
\usepackage{enumerate}
\usepackage{listings}
%\usepackage{fancyhdr}
%\cpic{<尺寸>}{<文件名>}}用于生成居中的图片。
\newcommand{\cpic}[2]{
\begin{center}
\includegraphics[scale=#1]{#2}
\end{center}
}
%\cpicn{<尺寸>}{<文件名>}{<注释>}用于生成居中且带有注释的图片,其label为图片名。
\newcommand{\cpicn}[3]
{
\begin{figure}[H]
\cpic{#1}{#2}
\caption{\color{red}#3\label{#2}}
\end{figure}
}

\crefname{equation}{}{}
\crefname{figure}{图}{图}
\crefname{footnote}{注释}{注释}
\crefname{table}{表}{表}
\title{实验D7 $\gamma$能谱测量实验}
\begin{document}
\maketitle
\begin{tabular}{|p{8em}|p{9em}|p{8em}|p{5em}|}
\hline
		实验方案 &实验记录  &分析讨论 &总成绩\\
		\hline
		        &          &          &  \\
	    \hline
	\hline 
	年级、专业: &17级物理学 &组号:& 6 \\
	\hline
	姓名:& 徐昊霆 &学号:&17353071  \\
	\hline
	日期:& \today &教师签名: &  \\
    \hline	
\end{tabular}
    \begin{enumerate}
 \item 实验报告由三部分组成:
 \begin{enumerate}
  \item[1)]预习报告:(提前一周)认真研读\textbf{\uline{实验讲义}},弄清实验原理;实验所需的仪器设备、用具及其使用(强烈建议到实验室预习),完成讲义中的预习思考题;了解实验需要测量的物理量,并根据要求提前准备实验记录表格(由学生自己在实验前设计好,可以打印)。预习成绩低于50\%者不能做实验{\color{red} (实验D2和D3需要提前一周的周四完成预习报告交任课老师批改,批改通过后,才允许做实验)}。

  \item[2)]实验记录:认真、客观记录实验条件、实验过程中的现象以及数据。实验记录请用珠笔或者钢笔书写并签名({\color{red}用铅笔记录的被认为无效})。{\color{red}保持原始记录,包括写错删除部分,如因误记需要修改记录,必须按规范修改。}(不得输入电脑打印,但可扫描手记后打印扫描件);离开前请实验教师检查记录并签名。
  \item[3)]分析讨论:处理实验原始数据(学习仪器使用类型的实验除外),对数据的可靠性和合理性进行分析;按规范呈现数据和结果(图、表),包括数据、图表按顺序编号及其引用;分析物理现象(含回答实验思考题,写出问题思考过程,必要时按规范引用数据);最后得出结论。
 \end{enumerate}
 \textbf{实验报告}就是预习报告、实验记录、和数据处理与分析合起来,加上本页封面。
 \item 每次完成实验后的一周内交\textbf{实验报告}。
 \item 除实验记录外,实验报告其他部分建议双面打印。
\end{enumerate}

    
\newpage
\tableofcontents
\newpage
\section{实验原理与方案}
\subsection{实验目的}
\begin{enumerate}
 \item[1.]测定$\gamma$谱仪的能量分辨率以及能量线性;
  \item[2.]测定已知源的$\gamma$能谱,并作能谱分析;
  \item[3.]随机数据处理;
\end{enumerate}


\subsection{仪器用具}
\begin{table}[H]
  \caption{$\gamma$能谱测量实验用具}
\cpic{0.4}{t1}
\end{table}

\subsection{实验安全注意事项}

使用真实放射源进行实验时,使用镊子放置和移除放射源,实验完毕后将放射源放回源库,并洗手。

\subsection{实验原理}
$\gamma$粒子是光子,其拉格朗日量为~\cite{electromagnetic}
\beq
\mathcal{L} = -\frac{1}{4}F_{\mu\nu}F^{\mu\nu}
\eeq
$\gamma$离子的波长比X射线更短。产生的方式有下面四种
1. 处于激发态的原子核向低能级跃迁时发射$\gamma$射线;

2. 正反粒子湮灭发射$\gamma$射线,如正负电子对湮灭产生0.511 MeV的$\gamma$光子;

3. 伴随核反应产生的$\gamma$射线,例如($\alpha$, $^{9}\mathrm{Be}$)→ ( $^{12}C*$,n)中子的产生过程;

4. 轫致辐射,当快电子与物质相互作用时,因受阻速度改变而辐射电磁辐射的现
象。轫致辐射的 γ 射线为连续能量谱。

高能$\gamma$光子的探测必须依赖于它与(探测器)物质相互作用的原理,而高能$\gamma$光子与物质相互作用的规律,决定了不可能(像可见光光子那样)将其能量一次性地转换为一个可测
量量(如电信号),而只能是通过多种、多次作用的能量损耗,将其能量转换为多个可测量
量(如光子数)。

\subsubsection{$\gamma$射线与物质的相互作用}
1. 光电效应
当能量$E_{\gamma}$的入射$\gamma$光子与物质中原子的束缚电子相互作用时,光子可以把全部能量转移给某个束缚电子,使电子脱离原子束缚而发射出去,光子本身消失,发射出去的电子称为
光电子,这种过程称为光电效应。发射出光电子的动能
\beq
E_e = E_{\gamma}-B_i
\eeq
$B_i$为束缚电子所在壳层的结合能。原子内层电子脱离原子后留下空位形成激发原子,其外
部壳层的电子会填补空位并放出特征 X 射线。$\gamma$射线与物质原子发生光电效应的总截面用汤姆逊散射截面公式表示,这一理论表明光电效应的反应截面正比于原子序数的5次方,反
比于$E_{\gamma}^{7/2}$ ,因此(1)重元素的光电效应截面比轻元素大;
(2)低能$\gamma$射线比高能$\gamma$射线发生
光电效应截面大;当$\gamma$射线能量低至接近电子的结合能时(即$E_\gamma \sim B_i$ ),光电效应的截面最大。

2. 康普顿效应
$\gamma$光子与自由静止的电子发生碰撞,而将一部分能量转移给电子,使电子成为反冲电子,$\gamma$光子被散射改变了原来的能量和方向。计算给出反冲电子的动能为
\beq\label{kinetic}
E_e = \frac{E^2_\gamma(1-\cos\theta)}{m_0c^2+E_\gamma(1-\cos\theta)}
\eeq
其中$m_0c^2$为电子静止质量,角度$\theta$是$\gamma$光子的散射角,如图~\ref{t2}所示。
\cpicn{0.6}{t2}{康普顿散射效应图}
由图看出反冲电子以$\varphi$角度出射,$\varphi$与$\theta$之间有如下关系
\beq
\cot \varphi = \left(1=\frac{E_\gamma}{m_0c^2}\right)\tan\frac{\theta}{2}
\eeq
由式~\ref{kinetic}可以看出,当$\theta = 180^{\circ}$时,反冲电子的动能$E_e$有最大值
\beq
E_{max} = \frac{E_\gamma}{1+\frac{m_0c^2}{2E_\gamma}}
\eeq
这说明康普顿效应产生的反冲电子的能量有一上限最大值,称为康普顿边界$E_c = E_{max}$。康普顿散射的反应截面与物质原子序数成正比。

3.电子对效应
\cpicn{0.4}{t3}{电子对效应图}
当 γ 光子能量大于 2 m 0 c 2 时,γ 光子从原子核旁边经过并受到核的库仑场作用,可能转
化为一个正电子和一个负电子,称为电子对效应。此时光子能量可表示为两个电子的动能与静止能量之和,如
\beq
E_\gamma = E_e^{+}+E_e^{-}+2m_0c^2
\eeq
其中$2m_0c^2 = 1.02MeV$。

电子对效应产生的正负电子在吸收物质中通过电离损失和轫致辐射损失能量。正电子很
快被慢化,与物质的电子发生电子湮灭,转换为两个光子。发生湮灭时正负电子动能为 0,
两个湮灭光子的总能量等于正负电子的静止能量(即 0.511 MeV)
。电子对湮灭后产生的两
个光子,在物质中会被吸收(吸收 1 个或者 2 个)或者发生康普顿散射。

综上所述,$\gamma$光子与物质相遇时,通过与物质原子发生光电效应、康普顿效应或电子对
效应而损失能量,其结果是产生次级带电粒子,如光电子、反冲电子或正负电子对。次级带
电粒子的能量与入射 γ 光子的能量直接相关,因此,可通过测量次级带电粒子的能量求得$\gamma$光子的能量。

二、闪烁体工作原理
闪烁体是闪烁探测器与粒子相互作用的部件,从化学成分来说可分为有机闪烁体(塑料
闪烁体)和无机闪烁体(NaI 闪烁体)。闪烁体具有吸收粒子能量转换成光子的本领,这也是
闪烁体最重要的特性之一。常用的表征闪烁体基本特性的参量有发光效率、发光时间和发光
衰减时间、光衰减长度。

发光效率是指闪烁体每吸收 1 MeV 的能量所能发射的光子数,常见闪烁体的发光效率
大约 103~105 /MeV,由发光效率乘以沉积在闪烁体中能量可简单估测入射粒子的光产额;
闪烁体产生很快,产生时间一般为 ps 量级;但是闪烁光子衰减很慢并且成指数规律衰
减,定义光子数从最大值衰减到原来的 1/e 所用的时间为衰减时间。一般无机闪烁体光衰减
时间为 ns 量级。

光衰减长度是描述闪烁光在闪烁体内传播的物理量,一般闪烁光强度随着传播距离成指
数衰减。光衰减长度分为本征衰减长度和技术衰减长度,前者取决于闪烁体内部特征(成分
和结构),后者是通过改变闪烁体的尺寸、表面光反射等技术手段来干预光衰减。

三、光电倍增管的工作原理。
光电倍增管是一种真空管,它由入射窗、光阴极、倍增系统(亦称倍增极、打拿极)和
阳极构成,如下图所示。入射窗的材料常为玻璃、石英或者蓝宝石等高透光材料;光阴极常
以发生光电效应概率较大的碱金属为主要材料;倍增系统由 8-13 个倍增极构成,以二次倍
增系数表征其倍增能力(每一级的倍增系数一般为为 3-6 倍,即经过一次倍增,二次电子数目增加至原来的 3-6 倍)
;阳极采用脱出功较大的材料制成,起到收集电子并输出信号的作
用。
\cpicn{0.4}{t4}{光电倍增管的构造图}
光电倍增管的工作原理:光子照射在光阴极后,吸收光子并发生光电效应,发射出光电
子,光电子在外加电场的作用下被加速打到倍增极并产生二次电子发射,二次电子又在外加
电场的作用下被加速打到下一级倍增极产生更过的二次电子发射,随着打拿极的增多,二次
电子的数目也得到倍增,最后被阳极吸收并产生电流或者电压信号,下图的分压电路会产生
处电压信号。



四、闪烁$\gamma$能谱仪
1. 闪烁谱仪的结构框图及各部分的功能
\cpicn{0.4}{t5}{闪烁谱仪的结构框图}
闪烁谱仪的结构框图示于图~\ref{t5}中,它可分为闪烁探头与高压、信号放大与多道分析
等两大部分。以下分别介绍各部分的功能。

(1) 闪烁探头与高压

闪烁探头包括闪烁体、光电倍增管、分压电路以及屏蔽外壳。实验中测量 γ 能谱多使用
无机闪烁体如 NaI (T1)晶体。闪烁体的功能是在次级带电粒子进入闪烁体时入射粒子损失部
分或全部能量,
,使闪烁体的原子和分子电离或激发,退激发的原子释放出低能光子。这些
低能光子被光导层引向加载高压的光电倍增管,并在其光敏阴极再次发生光电效应而产生光
电子,这些光电子经过一系列倍增极的倍增放大,从而使光电子的数目大大增加,最后在光
电倍增管的阳极上形成脉冲信号。脉冲数目是和进入闪烁体 γ 光子数目相对应的。而脉冲的
幅度与在闪烁体中产生的荧光光子数目成正比,从而和 γ 射线在闪烁体中损失的能量成正
比。整个闪烁探头应安装在屏蔽暗盒内以避免可见光对光电倍增管的照射而引起损坏。
上述转换的物理过程如表所示:
\cpicn{0.6}{t6}{信号转换流程图}

(2)信号放大与多道分析

由于探头输出的脉冲信号幅度很小,需要经过线性放大器将信号幅度按线性比例进行放
大,然后使用多道脉冲幅度分析器测量信号多道能谱。多道脉冲幅度分析器的功能是将输入
的脉冲按其幅度不同分别送入相对应的道址(即不同的存贮单元)中,通过软件可直接给出
各道址(对应不同的脉冲幅度)中所记录的脉冲数目,即得到了脉冲的幅度概率密度分布。
由于闪烁 γ 能谱仪输出的信号幅度与射线在晶体中沉积的能量成正比,也就得到了 γ 射线的
能谱。

2. NaI(TI)闪烁探测器测量的$\gamma$能谱形状
NaI(TI)闪烁$\gamma$能谱仪可测得$\gamma$能谱的形状,图~\ref{7-4}所示是典型$^{137}C_s$的 $\gamma$射线能谱图,图的纵轴代表各道址中的脉冲数目,横轴为道址,对应于脉冲幅度或$\gamma$射线的能量。
\cpicn{0.5}{7-4}{典型的$^{137}C_S$的$\gamma$射线能谱图}

从能谱图上看,有几个较为明显的峰,光电峰$E_e$,又称全能峰,其能量就对应$\gamma$射线
的能量$E_\gamma$。这是由于$\gamma$射线进入闪烁体后,由于光电效应产生光电子,其全部能量被闪烁体吸收。光电子逸出原子会留下空位,必然有外壳层上的电子跃入填
充,同时放出能量$E_z = B_i$的X射线,一般来说,闪烁体对低能X射线有很强的吸收作用,
这样闪烁体就吸收了$E_e+E_z$的全部能量,所以光电峰的能量就代表$\gamma$射线的能量,
对$^{137}C_s$,此能量为 0.662MeV。
$E_c$即为康普顿边界,对应反冲电子的最大能量。背散射峰$E_b$是由射线与闪烁体屏蔽
层等物质发生反向散射后进入闪烁体内而形成的光电峰,一般背散射峰很小。

结合本文第一部分$\gamma$射线与物质的相互作用原理,对$\gamma$射线在闪烁探测器中以及探测器周围发生的作用罗列在表中。表中展示了 γ 射线在闪烁体中各种相互作用过程对能谱分布的贡献及周围物质散射对谱形的影响。
\begin{table}[H]\label{6-2}
  \caption{$\gamma$射线在闪烁体中各种相互作用过程对能谱分布的贡献及周围物质散射对谱形的影响}
  \cpic{0.4}{6-2-1}
  \cpic{0.4}{6-2-2}
\end{table}

3. 谱仪的能量刻度和分辨率

(1)谱仪的能量刻度
闪烁谱仪测得的$\gamma$射线能谱的形状及其各峰对应的能量值由核素的衰变纲图所决定,是各核素的特征反映。但各峰被闪烁探测器采集后所对应的脉冲幅度是与工作
条件有关系的。如光电倍增管高压改变、线性放大器放大倍数不同等,都会改变各峰位在横
轴上的位置,也即改变了能量轴的刻度。因此,应用$\gamma$谱仪测定未知射线能谱时,必须先用
已知能量的核素能谱来标定$\gamma$谱仪。
由于能量与各峰位道址是线性的: $E_\gamma = kN+b$ ,因此能量刻度就是设法得到 $k$ 和 $b$。
例如选择$^{137}C_s$ 的光电峰 $E_\gamma =0.661 MeV$ 和$^{60}C_o$的光电峰$E_{\gamma 1}=1.17MeV$,如果对应$E_1=0.661 MeV$的光电峰位于$N_1$道,对应$E_2 = 1.17MeV$的光电峰位于$N_2$道,则有能量刻度
\beq
k = \frac{1.17-0.661}{N_2 - N_1} MeV \,\,\,b = \frac{(0.661+1.17)-k(N_1+N_2)}{2}MeV
\eeq
将测得的位置光电峰对应的道址$N$填入$E_\gamma = kN+b$即可得到对应的能量值。

(2)谱仪分辨率
$\gamma$能谱仪的一个重要指标是能量分辨率。由于闪烁谱仪测量粒子能量过程中,伴随着一
系列统计涨落过程,如$\gamma$光子进入闪烁体内损失能量、产生荧光光子、荧光光子在光阴极上打出光电子、光电子在倍增极上逐级倍增等等,这些统计涨落使脉冲的幅度服从统计规律而
有一定分布的。

定义谱仪能量分辨率$\eta$:
\beq
\eta = \frac{FWHM}{E_\gamma}\times 100\%
\eeq
其中FWHM(Full Width Half Maximum)表示选定能谱峰的半高全宽,$E_\gamma$为与谱峰对应的
$\gamma$光子能量, $\eta$表示闪烁谱仪在测量能量时能够分辨两条靠近的谱线的本领。目前一般的 NaI闪烁谱仪对$^{137}C_s$ 光电峰的分辨率在 10\%左右。对$\eta$的影响因素很多,如闪烁体、光电倍增管等等。

\subsection{实验前思考题}
1. 简述$\gamma$光子与物质的相互作用类型,及其产生的次级粒子与原入射光子的能量关系。

1. 光电效应
当能量$E_{\gamma}$的入射$\gamma$光子与物质中原子的束缚电子相互作用时,光子可以把全部能量转移给某个束缚电子,使电子脱离原子束缚而发射出去,光子本身消失,发射出去的电子称为
光电子,这种过程称为光电效应。发射出光电子的动能
\beq
E_e = E_{\gamma}-B_i
\eeq
$B_i$为束缚电子所在壳层的结合能。原子内层电子脱离原子后留下空位形成激发原子,其外
部壳层的电子会填补空位并放出特征 X 射线。$\gamma$射线与物质原子发生光电效应的总截面用汤姆逊散射截面公式表示,这一理论表明光电效应的反应截面正比于原子序数的5次方,反
比于$E_{\gamma}^{7/2}$ ,因此(1)重元素的光电效应截面比轻元素大;
(2)低能$\gamma$射线比高能$\gamma$射线发生
光电效应截面大;当$\gamma$射线能量低至接近电子的结合能时(即$E_\gamma \sim B_i$ ),光电效应的截面最大。

2. 康普顿效应
$\gamma$光子与自由静止的电子发生碰撞,而将一部分能量转移给电子,使电子成为反冲电子,$\gamma$光子被散射改变了原来的能量和方向。计算给出反冲电子的动能为
\beq\label{kinetic}
E_e = \frac{E^2_\gamma(1-\cos\theta)}{m_0c^2+E_\gamma(1-\cos\theta)}
\eeq
其中$m_0c^2$为电子静止质量,角度$\theta$是$\gamma$光子的散射角,如图~\ref{t2}所示。
\cpicn{0.6}{t2.png}{康普顿散射效应图}
由图看出反冲电子以$\varphi$角度出射,$\varphi$与$\theta$之间有如下关系
\beq
\cot \varphi = \left(1=\frac{E_\gamma}{m_0c^2}\right)\tan\frac{\theta}{2}
\eeq
由式~\ref{kinetic}可以看出,当$\theta = 180^{\circ}$时,反冲电子的动能$E_e$有最大值
\beq
E_{max} = \frac{E_\gamma}{1+\frac{m_0c^2}{2E_\gamma}}
\eeq
这说明康普顿效应产生的反冲电子的能量有一上限最大值,称为康普顿边界$E_c = E_{max}$。康普顿散射的反应截面与物质原子序数成正比。

3.电子对效应
\cpicn{0.4}{t3.png}{电子对效应图}
当 γ 光子能量大于 2 m 0 c 2 时,γ 光子从原子核旁边经过并受到核的库仑场作用,可能转
化为一个正电子和一个负电子,称为电子对效应。此时光子能量可表示为两个电子的动能与静止能量之和,如
\beq
E_\gamma = E_e^{+}+E_e^{-}+2m_0c^2
\eeq
其中$2m_0c^2 = 1.02MeV$。

电子对效应产生的正负电子在吸收物质中通过电离损失和轫致辐射损失能量。正电子很
快被慢化,与物质的电子发生电子湮灭,转换为两个光子。发生湮灭时正负电子动能为 0,
两个湮灭光子的总能量等于正负电子的静止能量(即 0.511 MeV)
。电子对湮灭后产生的两
个光子,在物质中会被吸收(吸收 1 个或者 2 个)或者发生康普顿散射。

综上所述,$\gamma$光子与物质相遇时,通过与物质原子发生光电效应、康普顿效应或电子对
效应而损失能量,其结果是产生次级带电粒子,如光电子、反冲电子或正负电子对。次级带
电粒子的能量与入射 γ 光子的能量直接相关,因此,可通过测量次级带电粒子的能量求得$\gamma$光子的能量。

2. 对探测器中的光电倍增管,射线的光电效应会对其测量产生影响吗?影响能有多大?

对于测量荧光光子的光子倍增管,理论上也会受到$\gamma$光子的光电效应作用,但是$\gamma$光子的界面远远小于荧光光子的界面,即荧光光子比高能$\gamma$光子更容易打出光电子;另一方面,荧光光子数远远多于$\gamma$光子数。

3. 有一单能$\gamma$源,能量为2MeV,根据$\gamma$与物质的相互作用及NaI(TI)闪烁能谱仪输出信号幅度的关系,预测能谱形状。

预测能谱形状如下
\cpicn{0.5}{7-4.png}{能谱形状}
有几个较为明显的峰,光电峰$E_e$,又称全能峰,其能量就对应$\gamma$射线
的能量$E_\gamma$。这是由于$\gamma$射线进入闪烁体后,由于光电效应产生光电子,其全部能量被闪烁体吸收。光电子逸出原子会留下空位,必然有外壳层上的电子跃入填
充,同时放出能量$E_z = B_i$的X射线,一般来说,闪烁体对低能X射线有很强的吸收作用,
这样闪烁体就吸收了$E_e+E_z$的全部能量,所以光电峰的能量就代表$\gamma$射线的能量,
对$^{137}C_s$,此能量为2MeV。
$E_c$即为康普顿边界,对应反冲电子的最大能量。背散射峰$E_b$是由射线与闪烁体屏蔽
层等物质发生反向散射后进入闪烁体内而形成的光电峰,一般背散射峰很小。

%%%%%%%%%%%%%%%%%%%%%%%%%%%%%%%%%%%%%%%%%%%%%%%%%%%%%%%%%
4. 闪烁谱仪是常用的$\gamma$能谱仪,简述闪烁谱仪的的结构图和各部分功能。

闪烁谱仪的结构框图及各部分的功能
\cpicn{0.4}{t5.png}{闪烁谱仪的结构框图}
闪烁谱仪的结构框图示于图~\ref{t5}中,它可分为闪烁探头与高压、信号放大与多道分析
等两大部分。以下分别介绍各部分的功能。

(1) 闪烁探头与高压

闪烁探头包括闪烁体、光电倍增管、分压电路以及屏蔽外壳。实验中测量$\gamma$能谱多使用
无机闪烁体如 NaI (T1)晶体。闪烁体的功能是在次级带电粒子进入闪烁体时入射粒子损失部
分或全部能量,
,使闪烁体的原子和分子电离或激发,退激发的原子释放出低能光子。这些
低能光子被光导层引向加载高压的光电倍增管,并在其光敏阴极再次发生光电效应而产生光
电子,这些光电子经过一系列倍增极的倍增放大,从而使光电子的数目大大增加,最后在光
电倍增管的阳极上形成脉冲信号。脉冲数目是和进入闪烁体$\gamma$光子数目相对应的。而脉冲的
幅度与在闪烁体中产生的荧光光子数目成正比,从而和$\gamma$ 射线在闪烁体中损失的能量成正
比。整个闪烁探头应安装在屏蔽暗盒内以避免可见光对光电倍增管的照射而引起损坏。
上述转换的物理过程如表所示:
\cpicn{0.6}{t6.png}{信号转换流程图}

(2)信号放大与多道分析

由于探头输出的脉冲信号幅度很小,需要经过线性放大器将信号幅度按线性比例进行放
大,然后使用多道脉冲幅度分析器测量信号多道能谱。多道脉冲幅度分析器的功能是将输入
的脉冲按其幅度不同分别送入相对应的道址(即不同的存贮单元)中,通过软件可直接给出
各道址(对应不同的脉冲幅度)中所记录的脉冲数目,即得到了脉冲的幅度概率密度分布。
由于闪烁 γ 能谱仪输出的信号幅度与射线在晶体中沉积的能量成正比,也就得到了 γ 射线的
能谱。

NaI(TI)闪烁探测器测量的$\gamma$能谱形状
NaI(TI)闪烁$\gamma$能谱仪可测得$\gamma$能谱的形状,图~\ref{7-4}所示是典型$^{137}C_s$的 $\gamma$射线能谱图,图的纵轴代表各道址中的脉冲数目,横轴为道址,对应于脉冲幅度或$\gamma$射线的能量。

从能谱图上看,有几个较为明显的峰,光电峰$E_e$,又称全能峰,其能量就对应$\gamma$射线
的能量$E_\gamma$。这是由于$\gamma$射线进入闪烁体后,由于光电效应产生光电子,其全部能量被闪烁体吸收。光电子逸出原子会留下空位,必然有外壳层上的电子跃入填
充,同时放出能量$E_z = B_i$的X射线,一般来说,闪烁体对低能X射线有很强的吸收作用,
这样闪烁体就吸收了$E_e+E_z$的全部能量,所以光电峰的能量就代表$\gamma$射线的能量,
对$^{137}C_s$,此能量为 0.662MeV。
$E_c$即为康普顿边界,对应反冲电子的最大能量。背散射峰$E_b$是由射线与闪烁体屏蔽
层等物质发生反向散射后进入闪烁体内而形成的光电峰,一般背散射峰很小。
结合本文第一部分$\gamma$射线与物质的相互作用原理,对$\gamma$射线在闪烁探测器中以及探测器周围发生的作用罗列在表 D6-2 中。表中展示了 γ 射线在闪烁体中各种相互作用过程对能谱分布的贡献及周围物质散射对谱形的影响。


3. 谱仪的能量刻度和分辨率

(1)谱仪的能量刻度
闪烁谱仪测得的$\gamma$射线能谱的形状及其各峰对应的能量值由核素的衰变纲图所决定,是各核素的特征反映。但各峰被闪烁探测器采集后所对应的脉冲幅度是与工作
条件有关系的。如光电倍增管高压改变、线性放大器放大倍数不同等,都会改变各峰位在横
轴上的位置,也即改变了能量轴的刻度。因此,应用$\gamma$谱仪测定未知射线能谱时,必须先用
已知能量的核素能谱来标定$\gamma$谱仪。
由于能量与各峰位道址是线性的: $E_\gamma = kN+b$ ,因此能量刻度就是设法得到 $k$ 和 $b$。
例如选择$^{137}C_s$ 的光电峰 $E_\gamma =0.661 MeV$ 和$^{60}C_o$的光电峰$E_{\gamma 1}=1.17MeV$,如果对应$E_1=0.661 MeV$的光电峰位于$N_1$道,对应$E_2 = 1.17MeV$的光电峰位于$N_2$道,则有能量刻度
\beq
k = \frac{1.17-0.661}{N_2 - N_1} MeV \,\,\,b = \frac{(0.661+1.17)-k(N_1+N_2)}{2}MeV
\eeq
将测得的位置光电峰对应的道址$N$填入$E_\gamma = kN+b$即可得到对应的能量值。

(2)谱仪分辨率
$\gamma$能谱仪的一个重要指标是能量分辨率。由于闪烁谱仪测量粒子能量过程中,伴随着一
系列统计涨落过程,如$\gamma$光子进入闪烁体内损失能量、产生荧光光子、荧光光子在光阴极上打出光电子、光电子在倍增极上逐级倍增等等,这些统计涨落使脉冲的幅度服从统计规律而
有一定分布的。

定义谱仪能量分辨率$\eta$:
\beq
\eta = \frac{FWHM}{E_\gamma}\times 100\%
\eeq
其中FWHM(Full Width Half Maximum)表示选定能谱峰的半高全宽,$E_\gamma$为与谱峰对应的
$\gamma$光子能量, $\eta$表示闪烁谱仪在测量能量时能够分辨两条靠近的谱线的本领。目前一般的 NaI闪烁谱仪对$^{137}C_s$ 光电峰的分辨率在 10\%左右。对$\eta$的影响因素很多,如闪烁体、光电倍增管等等。
\newpage
\section{实验步骤与记录}
\begin{tabular}{|p{8em}|p{8em}|p{8em}|p{8em}|}
	\hline 
	专业:     &Physics       &年级:      & 17     \\
	\hline
	姓名:& 徐昊霆 &学号:&17353071  \\
	\hline
	室温:&                    &实验地点 & 教学楼 \\
	\hline	
	学生签名: & & 评分: & \\
	\hline
	日期: & \today & 教师签名:&  \\
	\hline
\end{tabular}

本实验中使用虚拟核仿真信号源产生核脉冲信号,从而代替了放射源、探测器与高压
电源的使用;通用数据采集器使用多道分析功能,对信号源输出的核脉冲进行线性放大并
进行多道能谱测量与分析。通过软件控制虚拟核仿真信号源的电压和放射源的状态,可以得到相应的核脉冲信号,经过多道分析可以观察到相应的物理现象。虚拟核实验仪与真实
核实验仪对应的关系,如图~\ref{7-4}所示。
主要完成以下实验内容:
打开$\gamma$能谱实验软件,加载探测器高压,设置放射源源为$^{137}C_s$,打开多道分析仪软
件,记录$^{137}C_s$的能谱图,寻找并标记出光电峰和背散射峰的峰位,结合光电峰和背
散射峰的能量,定出谱仪的能量刻度,并通过光电峰的 FWHM 估算谱仪的能量分辨
率。

    
\subsection{实验中遇到的问题记录}
{\color{red} 寻峰过程中,电脑经常死机。应当及时保存数据,放置数据丢失。另外,建议优化电脑或者软件。}


\newpage
\section{分析与讨论}
\begin{tabular}{|p{7em}|p{7em}|p{7em}|p{7em}|}
	\hline 
	专业:     &Physics       &年级:      & 17     \\
	\hline
	姓名:& 徐昊霆 &学号:&17353071  \\
	\hline
	日期&  \today              & &  \\
	\hline	
	评分 & & 教师签名 & \\
	\hline
\end{tabular}
\subsection{谱仪能量刻度的标定}
在实验中我们放置放射源,加载高压后,打开多道分析仪软件,记录下了$^{137}C_S$的能谱图,并寻峰。得到如下图所示的图像。
\cpicn{0.4}{gamma137Cs}{$^{137}C_S$的$\gamma$能谱}
上面的测量中我们选取了1024个道址,其中利用软件自动的寻峰功能,我们发现$\gamma$能谱的全能峰在道址459.10处,半高宽为24.21个道址。通过查阅实验讲义~\cite{lec}知道,对应的能量值为
\beq
E_\gamma = 0.661 \mathrm{MeV}
\eeq
\cpicn{0.4}{gamma_peak2}{$^{137}C_S$的$\gamma$能谱的康普顿边界}
上面的测量中我们选取了1024个道址,其中利用软件自动的寻峰功能,我们发现$\gamma$能谱的康普顿边界在道址295.46处,半高宽为39.03个道址。
\cpicn{0.4}{gamma_peak1}{$^{137}C_S$的$\gamma$能谱的背景峰。}
上面的测量中我们选取了1024个道址,其中利用软件自动的寻峰功能,我们发现$\gamma$能谱的背散射峰道址129.16处,半高宽为41.06个道址。通过查阅实验讲义知道,背散射峰是由于$\gamma$射线在源及周围物质上发生电子对效应后的湮灭光子,其中一个湮灭光子从周围物质逃逸后被闪烁体吸收。对应的能量值为
\beq
E_b = 0.51\mathrm{MeV}
\eeq
因为道址和对应能量为线性关系,即假设能量和道址的关系为
\beq
E =kN+b
\eeq
所以我们可以得到
\beq
k = \frac{(0.661 - 0.51) \mathrm{MeV} }{495.10-129.16} = 4\times 10^{-4}\mathrm{MeV/d}
\eeq
上面的单位d代表“道”的意思~\footnote{这个新奇的单位是我瞎编的,起源于道的汉语拼音“dao”,在上式的中间过程里,d这个单位略去不写。}。使用误差传递公式,其实还可以计算出$k$的误差,$k$的误差来源与道址的误差,即道址的误差可以用半高全宽近似代替,进而算出$k$的误差。这里由于是虚拟实验,计算这个误差没有特别大的实际意义,就不计算了。

进而我们还可以把另外一个系数求出来
\beq
b = \left(0.661 - 4\times 10^{-4} \times 495.1\right) \mathrm{MeV}=0.463\mathrm{MeV}
\eeq
同理,我们也可以计算出$b$的误差,利用著名的误差传递公式~\footnote{如果读者十分关心误差传递公式是什么,请参见~\cite{error},因为按照实验报告的要求,如果使用了这些公式都要列出来,这里由于没有使用这些公式,就不列了。},$b$的误差也来源于道址的误差。如果采用我上面的计算公式去计算$b$的误差的话,还要加上$k$的误差。这里由于是虚拟实验,计算这个误差没有特别大的意义,就不计算了。最终,我们终于获取了{\color{red}能量和道址的数学表达式}
\beq
E = 4\times 10^{-4}\mathrm{MeV/d} \,\,\,\mathcal{N} + 0.463\mathrm{MeV}
\eeq
上面的$N$为道址,以道为单位~\footnote{而不是以$0.1$道这样的单位为单位};能量以MeV为单位。
\subsection{谱仪能量分辨率}
在实验中我们测量了三个峰,因为噪声信号的不同~\footnote{更严谨的说,是由于信噪比不同},我们测出的峰值的半高全宽就不同。比如在全能峰处,由于别的信号来源造成的在全能峰处附近信号较小,所以对全能峰的观测就影响较小。在实验中,我们利用公式
\beq
\eta = \frac{FWHM}{E_\gamma}\times 100\%
\eeq
其中FWHM是full width half maximum的缩写,在这个实验中,我们甚至不需要自己计算$\eta$,因为软件已经帮我们计算好了。对于全能峰,谱仪能量分辨率为$5.27\%$可见谱仪能量分辨率非常好。对于康普顿边界和背景散射峰来说,谱仪能量分辨率分别为$13.21\%$和$31.79\%$,这说明这两个峰附近由于其他因素导致的这个能量附近的光子比较多,在前面实验原理的表格中,我们发现确实在这两个附近光子有很多其他的来源,所以这个虚拟实验很好地验证了我们的理论。当然,真实情况与我们虚拟的情况仍然有差距,这也是虚拟实验的一个不足。
\subsection{$\alpha$粒子测量的实验}
在空闲时间,我们还做了$\alpha$粒子的测量实验,由于时间限制,我们只完成了实验内容一,即加大电压之后探求峰值和峰所对应的半高全宽的变化,在实验中我们发现半高全宽随着电压的增大而增大。由于本实验内容不属于必做,就不进行详细分析了。

  
\subsection{实验后思考题}
1. 解释的 137 Cs 的 $\gamma$ 射线能谱图中各个能峰的能量以及产生原因。
 137 Cs 的 $\gamma$ 射线能谱图中各个能峰的能量以及产生原因列在下表中。
 \begin{table}[H]
  \cpic{0.5}{6-2-1}
  \cpic{0.5}{6-2-2}
  \label{cause}
\end{table}

2. 能谱测量中,为减少统计误差,通常要尽量多的积累能谱数据(增大放射源强度或
者增加测量时间)
。某同学为了能在有限时间内积累足够的能谱数据,分别用两台
NaI 闪烁探测器同时测量137 Cs
的$\gamma$射线能谱图,那么该同学测量的这两组能谱数据能合起来使用吗?能的话,该怎样合起来使用?

能否结合起来用应该看该同学如何使用这两台仪器。如果两台仪器摆放的非常近,那么两台仪器探测到的数据就会有大量重复,所以这时把两台仪器的数据结合起来近似相当于把一台的数据加倍,没有任何实际意义。但是如果这位同学能够将两台仪器合理摆放,比如两台仪器测量放射源的两个正交的方向,这样可以认为两个正交方向的放射线没有任何显著的关系,这样的话就可以将数据直接相加。另外,如果两台仪器完全相同(包括信道的设置等),可以直接相加,如果两台仪器的不确定度不同,则需要根据每台仪器的误差将数据按找比例进行合成。(参见~\cite{error})

3.
137 Cs
能谱的峰位做一条垂直于横轴的线,能否看出这条线左侧低能区的总计数和
高能区的总计数,哪个多一些?结合$\gamma$粒子在闪烁体中的相互作用与其对能谱的影响分析,产生这一现象的原因。

在两个道址$N_1,N_2$之间的粒子数为
\beq
\Delta N = \sum_1^{2} \rho_i \Delta d
\eeq
也就是说,我们实际测得的能谱的纵坐标相当于能谱的密度。从上面测得的能谱图可以看到,全能峰左边的粒子数远远多于右边的粒子数。这是由于$\gamma$射线进入闪烁体后,由于光电效应产生光电子,其全部能量被闪烁体吸收。光电子逸出原子会留下空位,必然有外壳层上的电子跃入填充,同时放出能量$E_Z = B_i$的 X 射线,一般来说,闪烁体对低能 X 射线有很强的吸收作用,这样闪烁体就吸收了$E_e+E_z$的全部能量,所以光电峰的能量就代表$\gamma$射线的能量,对137Cs ,此能量为 0.662MeV。因此,没有明显的物理机制可以导致能量再高过$\gamma$射线本身的能量了,这也可以从能量守恒来解释。

4. 分析你测得的 137 Cs(0.662 MeV)的$\gamma$射线能谱图的全能峰,绘制能量-峰位图(横坐标为道数,纵坐标为对应能量)和不同$\gamma$能量的能量分辨率$\eta$图。

4-1 能量--峰位图和能量--道数关系

在实验中我们放置放射源,加载高压后,打开多道分析仪软件,记录下了$^{137}C_S$的能谱图,并寻峰。得到如下图所示的图像。
\cpicn{0.4}{gamma137Cs}{$^{137}C_S$的$\gamma$能谱}
上面的测量中我们选取了1024个道址,其中利用软件自动的寻峰功能,我们发现$\gamma$能谱的全能峰在道址459.10处,半高宽为24.21个道址。通过查阅实验讲义~\cite{lec}知道,对应的能量值为
\beq
E_\gamma = 0.661 \mathrm{MeV}
\eeq
\cpicn{0.4}{gamma_peak2}{$^{137}C_S$的$\gamma$能谱的康普顿边界}
上面的测量中我们选取了1024个道址,其中利用软件自动的寻峰功能,我们发现$\gamma$能谱的康普顿边界在道址295.46处,半高宽为39.03个道址。
\cpicn{0.4}{gamma_peak1}{$^{137}C_S$的$\gamma$能谱的背景峰。}
上面的测量中我们选取了1024个道址,其中利用软件自动的寻峰功能,我们发现$\gamma$能谱的背散射峰道址129.16处,半高宽为41.06个道址。通过查阅实验讲义知道,背散射峰是由于$\gamma$射线在源及周围物质上发生电子对效应后的湮灭光子,其中一个湮灭光子从周围物质逃逸后被闪烁体吸收。对应的能量值为
\beq
E_b = 0.51\mathrm{MeV}
\eeq
因为道址和对应能量为线性关系,即假设能量和道址的关系为
\beq
E =kN+b
\eeq
所以我们可以得到
\beq
k = \frac{(0.661 - 0.51) \mathrm{MeV} }{495.10-129.16} = 4\times 10^{-4}\mathrm{MeV/d}
\eeq
上面的单位d代表“道”的意思~\footnote{这个新奇的单位是我瞎编的,起源于道的汉语拼音“dao”,在上式的中间过程里,d这个单位略去不写。}。使用误差传递公式,其实还可以计算出$k$的误差,$k$的误差来源与道址的误差,即道址的误差可以用半高全宽近似代替,进而算出$k$的误差。这里由于是虚拟实验,计算这个误差没有特别大的实际意义,就不计算了。

进而我们还可以把另外一个系数求出来
\beq
b = \left(0.661 - 4\times 10^{-4} \times 495.1\right) \mathrm{MeV}=0.463\mathrm{MeV}
\eeq
同理,我们也可以计算出$b$的误差,利用著名的误差传递公式~\footnote{如果读者十分关心误差传递公式是什么,请参见~\cite{error},因为按照实验报告的要求,如果使用了这些公式都要列出来,这里由于没有使用这些公式,就不列了。},$b$的误差也来源于道址的误差。如果采用我上面的计算公式去计算$b$的误差的话,还要加上$k$的误差。这里由于是虚拟实验,计算这个误差没有特别大的意义,就不计算了。最终,我们终于获取了{\color{red}能量和道址的数学表达式}
\beq
E = 4\times 10^{-4}\mathrm{MeV/d} \,\,\,\mathcal{N} + 0.463\mathrm{MeV}
\eeq
上面的$N$为道址,以道为单位~\footnote{而不是以$0.1$道这样的单位为单位};能量以MeV为单位。

4-2谱仪能量分辨率

在实验中我们测量了三个峰,因为噪声信号的不同~\footnote{更严谨的说,是由于信噪比不同},我们测出的峰值的半高全宽就不同。比如在全能峰处,由于别的信号来源造成的在全能峰处附近信号较小,所以对全能峰的观测就影响较小。在实验中,我们利用公式
\beq
\eta = \frac{FWHM}{E_\gamma}\times 100\%
\eeq
其中FWHM是full width half maximum的缩写,在这个实验中,我们甚至不需要自己计算$\eta$,因为软件已经帮我们计算好了。对于全能峰,谱仪能量分辨率为$5.27\%$可见谱仪能量分辨率非常好。对于康普顿边界和背景散射峰来说,谱仪能量分辨率分别为$13.21\%$和$31.79\%$,这说明这两个峰附近由于其他因素导致的这个能量附近的光子比较多,在前面实验原理的表格中,我们发现确实在这两个附近光子有很多其他的来源,所以这个虚拟实验很好地验证了我们的理论。当然,真实情况与我们虚拟的情况仍然有差距,这也是虚拟实验的一个不足。



\bibliographystyle{siam}
\bibliography{cites}
\end{document}
