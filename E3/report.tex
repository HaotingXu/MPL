\documentclass{ctexart}
%\usepackage{CJKutf8}
%\setCJKmainfont{AR PL UKai CN}
%\setCJKmainfont{AR PL UKai}
\usepackage{geometry}
\usepackage{caption}
\usepackage{graphicx, subfig,float}
\geometry{a4paper,left=4cm,right=4cm}
\usepackage{appendix}
\usepackage{amsmath}
\usepackage{enumerate}
\usepackage{amssymb,color}
\usepackage[colorlinks,linkcolor=red,anchorcolor=blue,citecolor=blue]{hyperref}
\usepackage{slashed}
\usepackage{simplewick}
\usepackage{tikz}
\usepackage{tcolorbox}
\usepackage[T1]{fontenc}
%colors
\def\blacktext#1{{\color{black}#1}}
\def\bluetext#1{{\color{blue}#1}}
\def\redtext#1{{\color{red}#1}}
\def\darkbluetext#1{{\color[rgb]{0,0.2,0.6}#1}}
\def\skybluetext#1{{\color[rgb]{0.2,0.7,1.}#1}}
\def\cyantext#1{{\color[rgb]{0.,0.5,0.5}#1}}
\def\greentext#1{{\color[rgb]{0,0.7,0.1}#1}}
\def\darkgray{\color[rgb]{0.2,0.2,0.2}}
\def\lightgray{\color[rgb]{0.6,0.6,0.6}}
\def\gray{\color[rgb]{0.4,0.4,0.4}}
\def\blue{\color{blue}}
\def\red{\color{red}}
\def\green{\color{green}}
\def\darkgreen{\color[rgb]{0,0.4,0.1}}
\def\darkblue{\color[rgb]{0,0.2,0.6}}
\def\skyblue{\color[rgb]{0.2,0.7,1.}}
%%control
\def\be{\begin{equation}}
\def\ee{\nonumber\end{equation}}
\def\beq{\begin{equation}}
\def\eeq{\end{equation}}
\def\bea{\begin{eqnarray}}
\def\eea{\end{eqnarray}}
\def\bmat#1{\left(\begin{array}{#1}}
\def\emat{\end{array}\right)}
\def\bcase#1{\left\{\begin{array}{#1}}
\def\ecase{\end{array}\right.}
\def\bmini#1{\begin{minipage}{#1\textwidth}}
\def\emini{\end{minipage}}
\def\tbox#1{\begin{tcolorbox}#1\end{tcolorbox}}
\def\pfrac#1#2#3{\left(\frac{\partial #1}{\partial #2}\right)_{#3}}
%%symbols
\def\bropt{\,(\ \ \ )}
\def\sone{$\star$}
\def\stwo{$\star\star$}
\def\sthree{$\star\star\star$}
\def\sfour{$\star\star\star\star$}
\def\sfive{$\star\star\star\star\star$}
\def\rint{{\int_\leftrightarrow}}
\def\roint{{\oint_\leftrightarrow}}
\def\stdHf{{\textit{\r H}_f}}
\def\deltaH{{\Delta \textit{\r H}}}
\def\ii{{\dot{\imath}}}
\def\skipline{{\vskip0.1in}}
\def\skiplines{{\vskip0.2in}}
\def\lagr{{\mathcal{L}}}
\def\hamil{{\mathcal{H}}}
\def\vecv{{\mathbf{v}}}
\def\vecx{{\mathbf{x}}}
\def\vecy{{\mathbf{y}}}
\def\veck{{\mathbf{k}}}
\def\vecp{{\mathbf{p}}}
\def\vecn{{\mathbf{n}}}
\def\vecA{{\mathbf{A}}}
\def\vecP{{\mathbf{P}}}
\def\vecsigma{{\mathbf{\sigma}}}
\def\hatJn{{\hat{J_\vecn}}}
\def\hatJx{{\hat{J_x}}}
\def\hatJy{{\hat{J_y}}}
\def\hatJz{{\hat{J_z}}}
\def\hatj#1{\hat{J_{#1}}}
\def\hatphi{{\hat{\phi}}}
\def\hatq{{\hat{q}}}
\def\hatpi{{\hat{\pi}}}
\def\vel{\upsilon}
\def\Dint{{\mathcal{D}}}
\def\adag{{\hat{a}^\dagger}}
\def\bdag{{\hat{b}^\dagger}}
\def\cdag{{\hat{c}^\dagger}}
\def\ddag{{\hat{d}^\dagger}}
\def\hata{{\hat{a}}}
\def\hatb{{\hat{b}}}
\def\hatc{{\hat{c}}}
\def\hatd{{\hat{d}}}
\def\hatN{{\hat{N}}}
\def\hatH{{\hat{H}}}
\def\hatp{{\hat{p}}}
\def\Fup{{F^{\mu\nu}}}
\def\Fdown{{F_{\mu\nu}}}
\def\newl{\nonumber \\}
\def\vece{\mathrm{e}}
\def\calM{{\mathcal{M}}}
\def\calT{{\mathcal{T}}}
\def\calR{{\mathcal{R}}}
\def\barpsi{\bar{\psi}}
\def\baru{\bar{u}}
\def\barv{\bar{\upsilon}}
\def\qeq{\stackrel{?}{=}}
\def\torder#1{\mathcal{T}\left(#1\right)}
\def\rorder#1{\mathcal{R}\left(#1\right)}
\def\contr#1#2{\contraction{}{#1}{}{#2}#1#2}
\def\trof#1{\mathrm{Tr}\left(#1\right)}
\def\trace{\mathrm{Tr}}
\def\comm#1{\ \ \ \left(\mathrm{used}\ #1\right)}
\def\tcomm#1{\ \ \ (\text{#1})}
\def\slp{\slashed{p}}
\def\slk{\slashed{k}}
\def\calp{{\mathfrak{p}}}
\def\veccalp{\mathbf{\mathfrak{p}}}
\def\Tthree{T_{\tiny \textcircled{3}}}
\def\pthree{p_{\tiny \textcircled{3}}}
\def\dbar{{\,\mathchar'26\mkern-12mu d}}
\def\erf{\mathrm{erf}}
\def\const{\mathrm{constant}}
\def\pheat{\pfrac p{\ln T}V}
\def\vheat{\pfrac V{\ln T}p}
%%units
\def\fdeg{{^\circ \mathrm{F}}}
\def\cdeg{^\circ \mathrm{C}}
\def\atm{\,\mathrm{atm}}
\def\angstrom{\,\text{\AA}}
\def\SIL{\,\mathrm{L}}
\def\SIT{\, \mathrm{T}}
\def\SIkm{\,\mathrm{km}}
\def\SIyr{\,\mathrm{yr}}
\def\SIGyr{\,\mathrm{Gyr}}
\def\SIV{\,\mathrm{V}}
\def\SImV{\,\mathrm{mV}}
\def\SIeV{\,\mathrm{eV}}
\def\SIkeV{\,\mathrm{keV}}
\def\SIMeV{\,\mathrm{MeV}}
\def\SIGeV{\,\mathrm{GeV}}
\def\SIcal{\,\mathrm{cal}}
\def\SIkcal{\,\mathrm{kcal}}
\def\SImol{\,\mathrm{mol}}
\def\SIN{\,\mathrm{N}}
\def\SIHz{\,\mathrm{Hz}}
\def\SIm{\,\mathrm{m}}
\def\SIcm{\,\mathrm{cm}}
\def\SIfm{\,\mathrm{fm}}
\def\SImm{\,\mathrm{mm}}
\def\SInm{\,\mathrm{nm}}
\def\SImum{\,\mathrm{\mu m}}
\def\SIJ{\,\mathrm{J}}
\def\SIW{\,\mathrm{W}}
\def\SIkJ{\,\mathrm{kJ}}
\def\SIs{\,\mathrm{s}}
\def\SImus{\,\mathrm{\mu s}}
\def\SIkg{\,\mathrm{kg}}
\def\SIg{\,\mathrm{g}}
\def\SIK{\,\mathrm{K}}
\def\SImmHg{\,\mathrm{mmHg}}
\def\SIPa{\,\mathrm{Pa}}
\def\rad{\,\mathrm{rad}}


%\cpic{<尺寸>}{<文件名>}}用于生成居中的图片。
\newcommand{\cpic}[2]{
\begin{center}
\includegraphics[scale=#1]{#2}
\end{center}
}

%\cpicn{<尺寸>}{<文件名>}{<注释>}用于生成居中且带有注释的图片,其label为图片名。
\newcommand{\cpicn}[3]
{
\begin{figure}[H]
\cpic{#1}{#2}
\caption{#3\label{#2}}
\end{figure}
}
\title{实验E3 材料真空兼容性测试和等离子特性研究}
\begin{document}
\maketitle

\begin{tabular}{|p{8em}|p{8em}|p{8em}|p{5em}|}
\hline
		\large{实验方案} &\large{实验记录}  &\large{分析讨论} &\large{总成绩}\\
		\hline
		        &          &          &  \\
	    \hline
	\hline 
	年级、专业: &17级物理学 &组号:& 6 \\
	\hline
	姓名:& 徐昊霆 &学号:&17353071  \\
	\hline
	日期:& 2019.9.9 &教师签名: &  \\
    \hline	
        \end{tabular}
        
        1. 实验报告由三部分组成:
        
        1) 预习报告:(提前一周)认真研读实验讲义,弄清实验原理;实验所需的仪器设备、用具及其使用(强烈建议到实验室预习),完成讲义中的预习思考题;了解实验需要测量的物理量,并根据要求提前准备实验记录表格(由学生自己在实验前设计好,可以打印)。预习成绩低于50\%者不能做实验(实验D2和D3需要提前一周的周四完成预习报告交任课老师批改,批改通过后,才允许做实验)。
        
        2) 实验记录:认真、客观记录实验条件、实验过程中的现象以及数据。实验记录请用珠笔或者钢笔书写并签名(用铅笔记录的被认为无效)。保持原始记录,包括写错删除部分,如因误记需要修改记录,必须按规范修改。(不得输入电脑打印,但可扫描手记后打印扫描件);离开前请实验教师检查记录并签名。
        
    3) 分析讨论:处理实验原始数据(学习仪器使用类型的实验除外),对数据的可靠性和合理性进行分析;按规范呈现数据和结果(图、表),包括数据、图表按顺序编号及其引用;分析物理现象(含回答实验思考题,写出问题思考过程,必要时按规范引用数据);最后得出结论。
    实验报告就是预习报告、实验记录、和数据处理与分析合起来,加上本页封面。
    
    2. 每次完成实验后的一周内交实验报告。
    
    3. 实验报告建议双面打印。
\newpage
\tableofcontents
\newpage
\section{实验原理与方案}
\subsection{实验目的}
\par 
1. 学习基本的真空知识和技术,掌握真空的获得和测量方法。
\par
2. 通过真空气体放电实验,验证帕邢定律。

3. 了解四极杆质谱仪工作原理,掌握质谱仪的操作,进行真空系统检漏和真空环境分
析。

4. 研究等离子体特性,获得等离子体基本参数。

5. 使用质谱仪测试不同材料在真空中的脱气性质,了解材料的真空兼容性。

6. 深入探讨四极质谱仪工作原理。
\subsection{仪器用具}
上海宜准公司 VQP01真空平台,该装置由如下四部分构成。

\begin{tabular}{c|c|c|c}
	\hline
        编号 & 仪器名称 &数量& 主要参数(型号,测量范围,精度) \\
	\hline 
	1 &  放电管&1 & 用于实现空气(或氩气)的击穿和放电\\
        2 &  放电电源&1 &提供 0-1000V 的可调电压输出\\
        3 &  空气送气与调节系统&1& 包含气压测量装置\\
        4 &  击穿电压测量系统 &1&\\
	\hline
\end{tabular}

此外,为了保证高压电极处安全问题,特对高压电源实施继电器控制,只有真空计打开并且压强低于$1000\SIPa$ 时,高压电源才导通。
\subsection{实验安全注意事项}

1. 操作前请检查真空腔体是否密封,检查高压电源开关、分子泵电源开关是否断开,以及应急按钮是否断开。

2. 注意高电压电源使用安全。(高压电源受真空计控制,实验前请确认真空计是否通电;通电情况下请勿插拔高压电源后面板高压输出接口,切勿接触后侧电力控制部分;实验前请检查高压电源调节旋钮,务必置零;实验过程中请勿接触高压电源后面板以及高压电源内侧结构。)

3. 若实验中用到分子泵,需机械泵先抽真空压强低于 10 pa 以下才能开启分子泵电源。

4. 若实验中用到四极质谱仪,开启四极质谱仪时保证真空压强低于 $5.0\times 10^{-2}\SIPa$。

\subsection{实验原理}
\subsubsection{真空的获得和测量}
在给定空间内,气体压强低于一个大气压的气体状态,称之为真空。真空的获得就是人们常说的“抽真空”,即利用各种真空泵将被抽容器中的气体抽出,使该空间的压强低于一个大气压。真空测量是指用特定的仪器和装置,对某一特定空间内真空高低的测定,这种仪器或装置称为真空计(仪器、规管)。
\subsubsection{固体对气体的吸附及气体的脱附}
气体吸附就是固体表面捕获气体分子的现象,吸附分为物理吸附和化学吸附。其中物理吸附没有选择性,任何气体在固体表面均可发生,主要靠分子间的相互吸引力引起的。物理吸附的气体容易发生脱附,而且这种吸附只在低温下有效;化学吸附则发生在较高的温度下,与化学反应相似,气体不易脱附,但只有当气体和固体表面原子接触生成化合物时才能产生吸附作用。气体的脱附是气体吸附的逆过程。通常把吸附在固体表面的气体分子从固体表面被释放出来的过程叫做气体的脱附。
\subsubsection{气体放电、等离子体和帕邢定律}
气体放电的基本过程是利用外(电)场加速电子使
之碰撞中性原子(分子)来电离气体。等离子体由离子、电子以及未电离的中性原子(分子)
的集合组成,整体呈中性的物质状态。气体放电是产生等离子体的一种常见形式。帕邢定律是表征均匀电场气体间隙击穿电压、间隙距离和气压间关系的定律。帕邢定律的公式为
\beq
V_S = \frac{BPd}{\ln\left(\frac{APd}{\ln(1+1/\gamma)}\right)}
\eeq
上式中$A,B$在一定范围内是常数。$\gamma$为离子撞击阴极时所发生的电子发射的过程系数。帕邢定律在一定 $Pd$ 范围有效。气压过高或过高真空中,帕邢定律不适用。

帕邢曲线是根据帕邢定律的函数表达式所绘制的曲线,表达的物理意义为:击穿电压$U$是电极距离 $d$ 和气压 $P$ 乘积的函数。帕邢曲线的主要特点是:曲线在特定的 $Pd$ 值时,有最小的击穿电压。
\subsubsection{四极质谱仪}
四极杆上加有直流和射频交流分量电压(势),使得一定质量电荷比的离子可稳定的通过四极杆质量过滤器(离子能够稳定地通过四极电场),而不会撞上或逸出四极杆,可将离子根据质量电荷比进行过滤分类,归纳成质谱。这部分的详细原理讨论请见第三部分分析与讨论。
\newpage
\section{实验步骤与记录}
\begin{tabular}{|p{8em}|p{8em}|p{8em}|p{8em}|}
	\hline 
	专业:     &Physics       &年级:      & 17     \\
	\hline
	姓名:& 徐昊霆 &学号:&17353071  \\
	\hline
	室温:&                    &实验地点 & 教学楼 \\
	\hline	
	学生签名: & & 评分: & \\
	\hline
	日期: & 2019.9.9 & 教师签名:&  \\
	\hline
\end{tabular}
\subsection{实验内容}
\subsubsection{使用机械泵和分子泵获得高真空。}
\begin{enumerate}
\item 实验前请检查放电管与电源之间的电路连接是否可靠;电压调节旋扭是否最小位置;
气体流量调节旋扭是否置于最小位置。
\item 检查高压电源开关、分子泵电源开关以及应急开关是否处于断开状态。
\item 检查机械泵、分子泵、帕邢高压电源插头是否插上。
\item 检查帕邢实验高压电源与放电腔体 BNC 接头连接是否正常。
\item 检查完毕后,推上空气开关,打开电源总开关。
\item 查看电源按钮、电源指示灯是否正常(绿色)。
\item 开启机械泵,抽取真空至 2-3Pa,大约需要 10 分钟。
\item 等机械泵抽取压强至 10pa 以下后,闭合分子泵电源,等待 5 秒钟后,按分子泵电源面板上的绿色按钮,进行分子泵运行,继续抽取直至获得高真空
\end{enumerate}
\subsubsection{验证帕邢定律}
\begin{enumerate}
\item 测量两电极之间的实际间距$d$.
\item 检查放电管与电源之间的电路连接是否可靠;电压调节旋钮是否在最小位置;气体流量调节旋钮是否在最小位置。
\item 检查高压电源开关,分子泵电源开关是否处于断开状态.
\item 打开电源总开关
\item 开启机械泵,抽真空至$2-3 \SIPa$,大约需要等待十分钟。
\item  调节减压阀,使得流量计前气压在 0-1 大气压之间(指导教师准备)。   
\item 调节流量计的通气流量,至放电管内气压为 20Pa。   
\item 观察真空计数据并记录。   
\item 打开高压电源开关。   
\item 调节电源的电压输出,可以快速增至 200V,然后继续缓慢升高电压,直至气体发生击穿现象。读取击穿时的电压U。记录气压和电压的数值于表2-1。然后把电压降至 0V 以下,为下一次测量做好准备。在减小电压的过程中注意观察放电熄灭电压,并注意其与击穿电压的差别。   
\item 注意: 
1)增加电压的过程中,密切观察放电管电压表头和击穿电压表头的示数。 
2) 每个气压下,至少要重复 3 次测量,以三次击穿电压测量值之间的偏差不大于 15\%为成功测量,以得到可靠击穿电压。  
3) 在气压较高时,击穿前后,放电管的电压会有明显下降。接近击穿时的放电管电压为气体击穿电压。
\item 增加气体流量,使气压升高至 30Pa 左右,重复(10)的测量。   
\item 依次增加气体流量,每次增加 10Pa 左右,重复(10)。直至气压达到 100Pa。得到 8 组实验数据即可。
\item 减小气压回复至$20\SIPa$左右,重复(10).
\item 依次减小气压,每隔 2Pa 测量一组数据,直至 4Pa。测得 7-8 组数据即可。
\item 实验完毕后, 调节气体流量控制旋钮到最小位置,调节电压至最小值,依次关闭电压、机械泵、电源开关。
\end{enumerate}

实验表格见下页。
\newpage
两电极的实际距离$d = \, $

\begin{tabular}{|p{4em}|p{4em}|p{4em}|p{4em}|p{4em}|p{4em}|p{4em}|}
  \hline
  P[$\SIPa$] &\, &\, &\, &\, &\, &\, \\
  \hline
  U[V] &\,&\,&\,&\,&\,&\,\\
  \hline
  P[$\SIPa$] &\, &\, &\, &\, &\, &\, \\
  \hline
  U[V] &\,&\,&\,&\,&\,&\,\\
  \hline
  P[$\SIPa$] &\, &\, &\, &\, &\, &\, \\
  \hline
  U[V] &\,&\,&\,&\,&\,&\,\\
  \hline
  P[$\SIPa$] &\, &\, &\, &\, &\, &\, \\
  \hline
  U[V] &\,&\,&\,&\,&\,&\,\\
  \hline
  P[$\SIPa$] &\, &\, &\, &\, &\, &\, \\
  \hline
  U[V] &\,&\,&\,&\,&\,&\,\\
  \hline
  P[$\SIPa$] &\, &\, &\, &\, &\, &\, \\
  \hline
  U[V] &\,&\,&\,&\,&\,&\,\\
  \hline
  P[$\SIPa$] &\, &\, &\, &\, &\, &\, \\
  \hline
  U[V] &\,&\,&\,&\,&\,&\,\\
  \hline
  P[$\SIPa$] &\, &\, &\, &\, &\, &\, \\
  \hline
  U[V] &\,&\,&\,&\,&\,&\,\\
  \hline
  P[$\SIPa$] &\, &\, &\, &\, &\, &\, \\
  \hline
  U[V] &\,&\,&\,&\,&\,&\,\\
  \hline
  
\end{tabular}

\subsubsection{真空检漏和真空环境测量}
\par 开启机械泵并开启分子泵,在保证内部气压不大于$5\times 10^{-3}\SIPa$时,适当分别充进氩气、氦气和氖气。打开四极质谱仪并测量他们的光谱线。

\subsubsection{获得等离子体基本参数}
1、实验前检查真空腔体与电源之间的电路连接是否可靠;电压调节旋钮是否最小位置;气体流量调节旋钮是否最小位置。

2、检查高压电源开关以及分子泵电源开关是否处于断开状态,检查四级质谱仪电源是否连接,数据线是否连接电脑。

3、检查真空腔体气密性,检查应急开关是否处于断开状态。

4、打开电源总开关。

5、开启机械泵,抽真空至10Pa以下,大约需要十分钟。

6、机械泵抽取压强至10Pa以下后才能打开分子泵电源,等待5秒钟后,按分子泵电源面板上的绿色按钮,进行分子泵运行(此时分子泵转速会从450转降至68转左右进行自检,自检后开始从68转上升至450转)。

7、观察真空级显示屏,待分子泵抽取5.0E-2Pa以下再打开四级质谱仪。

8、继续抽取真空,待压强低于$10^{-3}\SIPa$后,可通过通气阀加入适量He、Ne、Ar等气体,便于通过软件观察各气体质谱图。(切记加入气体后压强勿超过5.0E-2Pa,若超过该压强,则需要继续抽气至低于该压强才能通过软件打开质谱仪灯丝,否则将严重损坏质谱仪灯丝)

9、打开测试软件VAccuRay3.0进行实验。

10、打开软件后点击右键,选择Device setup,出现以下界面:首次试验不用改变任何参数,直接点击OK.

11、然后点击任务栏灯丝选项,此时会出现“Be Sure pressure Less Than 5.0E-2Pa”提示,观察真空计确定真空度低于该提示值是才能点击确定,否则将损坏四级质谱仪的灯丝,发生危险。

12、点击确认后会出现Spectrum质谱图,此时即可通过各种粒子的相对分子质量确定气体的波峰。

13、右键,选择Calculate Width,进行波峰计算,并记录相似数值。

14、记录数据后返回质谱图界面,右键,选择Device setup,改变参数resolution为30、40、50…100
15、点击OK,重复(10)(11)(12)(13)步骤,并记录数据。

16、之后改变数据Emission、e Volt等参数进行上述(10)(11)(12)(13)步骤实验并记录数据。

17、实验结束后,先将分子泵电源面板处按红色按钮,待转速降为0时,才能断开分子泵电源开关才能断开机械泵开关,顺序切勿颠倒,否则将损坏分子泵,最后关闭总电源。

18、整理数据,进行总结

%\subsubsection{测试不同材料脱气性质}
\subsubsection{研究四极电场特性及其中离子运动方程,探讨四极质谱仪工作原理}
这部分是纯理论工作,请见分析与讨论。
\subsection{实验中遇到的问题记录}
\par 1. 实验仪器特别好,分辨率为0也可以扫出占比很少的水蒸气、空气。
\par 2. 软件或者仪器问题,导致了曲线坐标轴整体平移。

\subsection{实验原始数据}
\cpicn{0.25}{original_1.jpg}{实验原始数据记录表格:$d = 5\SIcm$时帕邢定律的验证}
\cpicn{0.25}{original_2.jpg}{实验原始数据记录表格:$d = 7.05\SIcm$时帕邢定律的验证}
\newpage
\section{分析与讨论}
\begin{tabular}{|p{8em}|p{8em}|p{8em}|p{8em}|}
	\hline 
	专业:     &Physics       &年级:      & 17     \\
	\hline
	姓名:& 徐昊霆 &学号:&17353071  \\
	\hline
	日期&     2019.9               & &  \\
	\hline	
	评分 & & 教师签名 & \\
	\hline
\end{tabular}
\subsection{真空的获得}
在实验中,我们学习了如何使用真空平台。我们先打开总电源,之后打开机械泵,等待压强小于$10\SIPa$,之后我们打开了分子泵,经过半个小时的等待,我们发现可以抽到$10^{-5}\SIPa$的量级。
\subsection{帕邢定律的验证}
帕邢定律是有关击穿电压的定律。根据我们的参考资料,帕邢定律只在一定的气压范围内成立,对于太低的气压是不成立的。这一点可以从帕邢定律的数学表达式子看出来,帕邢定律可以写成
\beq\label{eq:paxing}
V_S = \frac{BPd}{\ln\left(\frac{APd}{\ln(1+1/\gamma)}\right)}
\eeq
如果我们定义
\bea
B^{\prime} &=& Bd  \label{eq:A} \\
A^{\prime}&=& \frac{Ad}{\ln(1+1/\gamma))} \label{eq:B}
\eea
这样我们就把帕邢定律写成
\beq
V_S = \frac{B^{\prime}P}{\ln(A^{\prime}P)}
\eeq
所以可见,如果按照帕邢定律,$V_S$将在两个地方发散,第一个地方是$P = 0$,击穿电压趋近于负无穷,而第二个地方则是$A^{\prime}P_0 = 1$。也就是说,按照帕邢定律在$A^{\prime}P_0 = 1$时,电压无论达到多少都不足以击穿气体,这显然与物理相违背。所以我们得出,在$P<P_0$时,帕邢定律失效。图~\ref{general.jpeg}画出了帕邢定律的一条一般曲线,由此可见击穿电压在低气压处的不合理性。
\cpicn{0.5}{general.jpeg}{\color{red} 帕邢定律曲线的形状。}
\par  另外,由于在原来公式中~\ref{eq:paxing},$A$与$B$只是常数,所以在数据处理时,我们只需要拟合$A^{\prime},B^{\prime}$。因为帕邢定律中还有与距离的关系,我们只需要改变距离,这时我们拟合出$A^{\prime}(d)$,就可以验证帕邢定律对于距离变化的情况的正确性。
\subsubsection{拟合方法}
首先,在拟合之前,我们要考虑到测量值本身的误差。由于我们对一个点反复测量了三次,所以我们得到平均值的标准偏差为
\beq
\sigma_V = \sqrt{\frac{\sum_1^3 (V-<V>)^2}{3\times 2}}
\eeq
其中$<V>$为三次测量的平均值,这一误差通过误差棒的形式体现在下面的图中。

我使用了MCMC方法(蒙特卡洛马尔科夫链)进行数据拟合。如果我们有观测量$\mu_i$,它的协方差矩阵是$\rm{cov}_{ij}$,理论模型(理论模型由一系列参数$\theta$决定,在这里是帕邢定律的$A,B$)决定了观测量的理论值$\mu_{\rm th}$。  则定义$\chi^2$为
\beq
\chi^2 = (\mu - \mu_{\rm th}(\theta) )^{T} \rm{cov} (\mu - \mu_{\rm th}(\theta) )
\eeq
MCMC方法就是通过随机行走的方法不断改变$\theta$的值,使得上述$\chi^2$最小。
\subsubsection{验证帕邢定律关于压强$P$的关系}
通过上述的方法,结合我们的实验数据,我们得到如图~\ref{paxing.jpg}所示的图像。在图像中我们已经画出了误差棒。可见,气压较低的情况下得到的误差相对较大,而气压很高的时候误差则很小。得到$A^{\prime},B^{\prime}$的最佳可能值及误差为
\bea
A^{\prime} &=& 0.2133 \pm 0.00735\\
B^{\prime} &=& 12.9 \pm 1.5
\eea
\cpicn{0.5}{paxing.jpg}{板间距$d = 5\SIcm $时,击穿电压与压强的关系,红色的曲线为使用MCMC方法拟合出来的帕邢定律的最佳曲线}

进而,通过加入我们测到的板间距$d = 5\SIcm$,可以得到帕邢定律中系数$A,B$
\bea
\frac{A}{\ln(1+1/\gamma)} &=& 4.3 \pm 0.1 \, /(\SIPa \cdot\SIm) \\
B &=& 258 \pm 30 \,\SIV /(\SIPa\cdot \SIm)
\eea
不单单可以从图上看到气体的击穿电压与帕邢定律符合的很好,使用统计学上的$R^2$(coefficient of determination)来衡量曲线拟合的准确性也可以证实这一点。经过计算,上面拟合的$R^2$为
\beq
R^2 = 0.82
\eeq
可见气体的击穿电压与气压的关系与帕邢定律符合的非常好。

\subsubsection{验证帕邢定律与距离的关系}
如果要严谨地验证这一关系,需要多次改变不同的距离,并且在每个距离下不断改变气压来做实验。由于时间有限,我们只能验证两组。也就是说,通过一次实验我们得到了帕邢定律中系数$A^{\prime}(d_1),B^{\prime}(d_1)$,而通过另一次改变距离的实验我们得到了系数$A^{\prime}(d_2),B^{\prime}(d_2)$,由于式~\ref{eq:A}, ~\ref{eq:B},我们应该得到
\bea
\frac{B^{\prime}(d_2)}{B^{\prime}(d_1)} &=& \frac{d_2}{d_1} \\
\frac{A^{\prime}(d_2)}{A^{\prime}(d_1)} &=& \frac{d_2}{d_1} 
\eea
我们采取同样的方法对$d_2 = 7.05\SIcm$时气体击穿电压与压强的曲线进行拟合,得到的曲线如图所示。
\cpicn{0.5}{paxing_d2.jpg}{\color{red} $d_2=7.05\SIcm$时击穿电压和空气压强的关系。由于时间仓促,数据点较少,所以拟合结果不是很好。但是也可以从中看出帕邢定律曲线的形状。}
同样地,计算得到$A,B,R^2$的值为
\bea
A^{\prime} &=& 0.33\pm 0.01 \\
B^{\prime} &=& 15.2 \pm 1.2 \\
R^2 &=& 0.979 \label{R2:2}
\eea
由式子~\ref{R2:2}可知,拟合结果令人满意,但是显然,这次得到的误差较大,这是由于数据点较少导致的。经过上面的讨论,我们计算得到
\bea
\frac{B^{\prime}(d_2)}{B^{\prime}(d_1)} &=& 1.5 \pm 0.3 \\
\frac{A^{\prime}(d_2)}{A^{\prime}(d_1)} &=& 1.2 \pm 0.5
\eea
其中,上面的误差由下面的公式算出,设$f(x_1,x_2)$则$f$的误差为
\beq
\sigma_f = \sqrt{\left(\partial_1 f \sigma_1\right)^2+\left(\partial_2 f \sigma_2\right)^2}
\eeq
由于在实际中,距离的关系是$d_2/d_1 = 1.4$可见真值都落在一定的置信区间内。可以说,在一定程度上证实了帕邢定律对于距离的正确性。但是由于第二次测量比较粗糙,最终就导致平均的估计有所偏差,为了改善这一偏差,再次改变距离进行实验是有必要的。
\subsection{真空环境检漏}
\par 开启机械泵并开启分子泵,在保证内部气压不大于$5\times 10^{-3}\SIPa$时,适当分别充进氩气、氦气和氖气。打开四极质谱仪并测量他们的光谱线。由于仪器原因,我们测量得到的曲线在横轴有一定的偏移。

\cpicn{0.3}{He.PNG}{\color{red} 充入氦气后,分压与荷质比关系曲线。}
\par 在气压环境大约为$5\times 10^{-4}\SIPa$情况下,向真空管内充入一定量的氦气,得到的气压分量与电荷质量比的曲线如图所示。曲线的最左端的峰是由于技术原因出现的,并不是真实的气体。由元素周期表可以知道,He的相对分子质量为2,在曲线可以看到大概在4处有明显的峰,这是由于软件本身错误而导致的坐标轴平移。在荷质比为17,19处,我们还看到两个小的峰,这实际上是相对分子质量为14,16的氮气和氧气和相对分子质量为18的水蒸气。水蒸气是来自于气体本身的不纯净。然而这些峰并不是精确地落在我们所预测的点,这是由于软件的问题导致的,实际上如果将坐标轴整体平移,则可以得到精准的曲线。另外由于仪器原因,安装在电脑上的软件无法读取真空腔内真实的气压,而是显示的默认值,所以我们得到的分压(y轴)并不是真是的分压,实际上如果做归一化$p = P/P_0$,其中$P_0$为系统默认值,假设荷质比为$u$,则有关系
\beq
\int p(u) du =1
\eeq
其中$p(u)$是单位荷质比宽度粒子出现的概率。所以如果对于一定的区间,粒子的百分比就为
\beq
\int_a^b p(u) du
\eeq
也就是说,曲线下面的面积代表粒子的百分数。可以看到,我们的仪器十分灵敏,即使是占比很小的、可能来源于充入气体本身的水蒸气和空气,我们都能探测到。

\cpicn{0.3}{Ne.PNG}{\color{red} 充入氖气后,分压与荷质比关系曲线。}
\par 在气压环境大约为$5\times 10^{-4}\SIPa$情况下,向真空管内充入一定量的氖气,得到的气压分量与电荷质量比的曲线如图所示。曲线的最左端的峰是由于技术原因出现的,并不是真实的气体。由元素周期表可以知道,Ne的相对分子质量为10,在曲线可以看到大概在20处有明显的峰,这是由于大多数氖原子都都有两个电子电离,从而氖的原子核带两个正电荷的缘故,所以探测到的荷质比为20。在荷质比为14,18处,我们还看到两个小的峰,这实际上是相对分子质量为14,16的氮气和氧气和相对分子质量为18的水蒸气。水蒸气是来自于气体本身的不纯净。另外由于仪器原因,安装在电脑上的软件无法读取真空腔内真实的气压,而是显示的默认值,所以我们得到的分压(y轴)并不是真是的分压,实际上如果做归一化$p = P/P_0$,其中$P_0$为系统默认值,假设荷质比为$u$,则有关系
\beq
\int p(u) du =1
\eeq
其中$p(u)$是单位荷质比宽度粒子出现的概率。所以如果对于一定的区间,粒子的百分比就为
\beq
\int_a^b p(u) du
\eeq
也就是说,曲线下面的面积代表粒子的百分数。可以看到,我们的仪器十分灵敏,即使是占比很小的、可能来源于充入气体本身的水蒸气和空气,我们都能探测到。我们在相对分子质量为32、36处也能看到一些峰,同样地,这是由于水蒸气分子和氮气氧气分子电离出两个电子而造成的。

\cpicn{0.3}{Ar.PNG}{\color{red} 充入氩气后,分压与荷质比关系曲线。}
\par 在气压环境大约为$5\times 10^{-4}\SIPa$情况下,向真空管内充入一定量的氩气,得到的气压分量与电荷质量比的曲线如图所示。曲线的最左端的峰是由于技术原因出现的,并不是真实的气体。由元素周期表可以知道,Ar的相对分子质量为40,在曲线可以看到大概在40处有明显的峰,至于那个峰不是严谨的在40处,有一点偏移,这是由于软件原因造成的。在荷质比为14,16,18处,我们还看到三个小的峰,这实际上是相对分子质量为14,16的氮气和氧气和相对分子质量为18的水蒸气。水蒸气是来自于气体本身的不纯净。另外由于仪器原因,安装在电脑上的软件无法读取真空腔内真实的气压,而是显示的默认值,所以我们得到的分压(y轴)并不是真是的分压,实际上如果做归一化$p = P/P_0$,其中$P_0$为系统默认值,假设荷质比为$u$,则有关系
\beq
\int p(u) du =1
\eeq
其中$p(u)$是单位荷质比宽度粒子出现的概率。所以如果对于一定的区间,粒子的百分比就为
\beq
\int_a^b p(u) du
\eeq
也就是说,曲线下面的面积代表粒子的百分数。可以看到,我们的仪器十分灵敏,即使是占比很小的、可能来源于充入气体本身的水蒸气和空气,我们都能探测到。我们在相对分子质量为32、36处也能看到一些峰,同样地,这是由于水蒸气分子和氮气氧气分子电离出两个电子而造成的。




\subsection{四极质谱仪原理的深入研究}

\par 本节深入探讨四极质谱仪的工作原理。
\subsubsection{Mathieu 方程}
\par 根据提供的参考资料\cite{ref4},我们知道离子在四极质谱仪中的四极场中运动,它的$x$方向(垂直于离子的入射方向)的运动方程为
\beq
\frac{d^2x}{dt^2}+\left(\frac{2eU}{mr_0^2}+\frac{2eV\cos\Omega t}{mr_0^2}\right)x=0
\eeq
如果作替换
\bea
y &=&x \\
 t &=& 2z/\Omega \\
\lambda &=& \frac{2eU}{mr_0^2}\frac{\Omega^2}{4} \\
q &=& -\frac{eV\cos\Omega t}{mr_0^2}\frac{\Omega^2}{4} 
\eea
则我们得到标准的Mathieu方程
\beq \label{eq:ma}
\frac{d^2y}{dz^2}+(\lambda-2q\cos 2z)y = 0.
\eeq
\subsubsection{基本解的概念}
在我们处理Mathieu方程之前,我们先考虑更为一般的方程
\beq \label{eq:general}
\frac{d^2y}{dz^2}+(\lambda-\phi(z))y = 0
\eeq
其中$\phi(z)$是周期为$\omega$的函数。

假设$f(z)$和$g(z)$是满足方程~\ref{eq:general}的解,且分别满足下列初值条件
\bea \label{eq:condition}
f(0) = 1, &f^{\prime}(0)=0, \\
g(0) = 0, &g^{\prime}(0) = 1 
\eea
因为这两个解均满足微分方程和上面的条件,故很容易得到等式
\beq
f(z)g^{\prime}(z) - f^{\prime}(z)g(z) = \rm{const} = 1
\eeq
因此,$f(z), g(z)$是两个线性无关的解,我们称作\textbf{基本解}。由于$\phi(z)$是周期为$\omega$的周期函数,所以$f(z\pm \omega)$和$g(z \pm \omega)$也是方程~\ref{eq:general}的解。又根据微分方程理论,一个二阶微分方程至多有两个线性无关的解,所以微分方程的任何一个解都可以表示成基本解的线性组合,于是有
\beq
f(z\pm \omega) = A_{\pm} f(z) + B_{\pm} g(z)
\eeq
可以利用条件~\ref{eq:condition}定出上面式子的系数。
\beq\label{eq:4}
f(z\pm \omega) = f(\pm \omega) f(z) + f^{\prime}(\pm \omega) g(z)
\eeq
这里值得注意的是,如果微分方程的某个系数具有周期性,那么不能保证它的解也具有同样的周期性。类似地,对于$g(z)$有
\beq\label{eq:5}
g(z\pm \omega) = g(\pm \omega) f(z) + g^{\prime}(\pm \omega) g(z)
\eeq
如果我们进一步假定$\phi(z)$是偶函数(在Mathieu方程中情况确实是这样的),则显然根据对称性,$f(-z), g(-z)$也是方程~\ref{eq:general}的解。利用方程~\ref{eq:condition},得到
\bea 
f(-z) &=& f(z)\\
g(-z) &=& -g(z)
\eea
于是我们得到结论,当$\phi(z)$为偶函数时,由条件~\ref{eq:condition}所得到的基本解$f(z), g(z)$,前者是偶函数,后者是奇函数。因此,方程~\ref{eq:general}不能同时拥有两个线性无关的偶函数解或者两个线性无关的奇函数解。在上述条件成立的情况下,方程~\ref{eq:4}和方程~\ref{eq:5}应该改写为
\beq\label{eq:8}
f(z\pm \omega) = f( \omega) f(z) \pm f^{\prime}( \omega) g(z)
\eeq
\beq\label{eq:9}
g(z\pm \omega) =\pm g( \omega) f(z) + g^{\prime}( \omega) g(z)
\eeq
取~\ref{eq:8}和~\ref{eq:9}的负号,利用条件~\ref{eq:condition}并假设$g(\omega)\neq0$,我们得到一个十分重要的结果
\beq\label{eq:im}
f(\omega) = g^{\prime}(\omega)
\eeq
\subsubsection{方程的Floquet解}

\par 系数$\phi(z)$是周期函数的方程 
\beq 
\frac{d^2y}{dz^2}+(\lambda-\phi(z))y = 0
\eeq
的解$\phi(z)$如果具有如下性质
\beq
y(z+\omega) = \sigma y(z)
\eeq
其中$\sigma$是与$z$无关的常数,$\omega$是函数$\phi(z)$的周期,则$y(z)$称为Floquet解。取$z=0$有
\beq
y(\omega) = \sigma y(0), \, y^{\prime} = \sigma y^{\prime}(0)
\eeq
现在我们看一下什么条件下,Floquet解存在。假设
\beq
y(z) = Af(z) + Bg(z)
\eeq
则得到
\bea
    [f(\omega)-\sigma]A +g(\omega) B &=& 0\\
    f^{\prime}(\omega) A + [g^{\prime}(\omega)-\sigma]B &=& 0
\eea
这是一个关于$A,B$的齐次方程,如果他们有非平凡解,则要求系数行列式为0。如果重写系数$\sigma = \exp(i\nu \omega)$,则有
\beq
\cos \nu \omega = \frac{1}{2}\left[f(\omega)+g^{\prime}(\omega)\right]
\eeq
利用式子~\ref{eq:im},则有
\beq\label{eq:important}
\cos\nu\omega = f(\omega)
\eeq
方程的Floquet解总能写成下面的形式
\beq
y(z) = \exp(i\nu \omega)u(z)
\eeq
其中$u(z)$是周期为$\omega$的周期函数,证明略。
\subsubsection{Floquet解与量子力学中的Bloch定理}
\par 写到这里,我不禁联想到量子力学中固体中的价带结构(Band Structure)。我发现,如果不看一些系数,上面讨论的一般情况的方程~\ref{eq:general}恰好是一个含有周期势场的薛定谔方程!这时薛定谔方程的解满足著名的Bloch定理
\beq
\psi(z+a) = e^{iKa}\psi(a)
\eeq
这完全是对上面讨论的Floquet解的对应。因此,处理上面的方程实际上还可以直接使用量子力学的办法。即定义位移算符
\beq
Df(x) = f(x+a)
\eeq
通过直接的证明可以得到,位移算符与哈密顿算符对易,故可以取一个波函数$\psi$,使得它同时是位移算符和哈密顿算符的本征函数
\beq
D\psi = e^{iKa} \psi(x) = \psi(x+a)
\eeq
这样,就用量子力学的语言印证了上面的讨论。进一步地,如果保证这个方程有稳定解,则会导致有价带结构(Band Structure)的出现。也就意味着,能量必须间隔一段间隔。在这个问题中,能量的取值取决于参数$\lambda,q$,可以断言,由于势能场是周期的,则能级一定会存在类似的间隔。四极质谱仪正是利用这个特性来筛选掉参数处在这些空穴的离子。下面我将具体地给出稳定解的条件。

\subsubsection{Mathieu方程的周期解}
\par 数学家已经证明(由于在这里叙述篇幅过长,故略去),对于Matheiu方程~\ref{eq:ma}中的参数$\lambda,q$,当他们满足一定条件时,该方程有周期为$\pi$或者$2\pi$的周期解,利用周期性和式~\ref{eq:important}
\beq
\cos\nu \pi = f(\pi;\lambda,q)
\eeq
显然,稳定解存在的条件是
\beq\label{eq:condition}
|f(\pi;\lambda,q)|<1
\eeq
否则,$\nu$将取虚数,也就意味着Floquet解发散。进一步地分析得到,函数$f(z)$实际上是Mathieu函数$ce_m(z)$,于是通过计算机就可以解出上述方程参数$\lambda,q$的取值范围。要获取更多关于Mathieu函数的性质,读者可以查看~\cite{ma_func}~\cite{special_func}。至此,我们没有具体地讨论复杂的Mathieu函数的性质,而理解了四极质谱仪的工作原理。




\subsection{实验后思考题}
1.为什么机械泵先抽真空压强低于 10 pa 以下才能开启分子泵电源?

分子泵的工作原理是产生定向的分子流,而这一分子流只有在分子间的平均间距很大时才能够正常工作。分子平均自由程$l_f$的定义为
\beq
n l_f A = 1
\eeq
其中$A$是分子的截面积。上述定义意味着,分子扫过$l_f A$大小的体积可以遇到一个分子,所以我们有
\beq
l_f = \frac{1}{nA}
\eeq
所以可见,在分子的截面积固定时,分子的平均自由程与分子数密度成反比。由于我们考虑的气体都很稀薄,气体的状态方程可以近似为
\beq
P = nkT
\eeq
其中$n$是分子数密度,$k$是玻尔兹曼常数,$T$是温度。所以当温度固定时,压强越小,分子的数密度越小。由此可见,需要将气压降低到一定程度才能使得分子泵正常工作。

2.为什么开启四极质谱仪时保证真空压强低于 $5.0\times 10^{-2}\SIPa$ ?

因为气体数密度过大的环境会损伤灯丝。实际上,灯丝的作用就是发射不同波段的电磁波,使得气体分子中的电子获得足够高的能量而电离,从而获得原子。如果灯丝在点燃状态,而灯丝的附近的分子数密度非常大,则单位时间内就会有很多的分子电离,而带电粒子与灯丝产生的碰撞作用会大大减少灯丝的寿命,甚至直接使得灯丝烧坏。

3.为什么气体放电产生的等离子体能发出辉光?

辉光放电是利用产生的电子将中性原子或分子激发,而被激发的粒子由激发态降回基态时会以光的形式释放出能量。

辉光放电时,在放电管两极电场的作用下,电子和正离子分别向阳极、阴极运动,并堆积在两极附近形成空间电荷区。因正离子的漂移速度远小于电子,故正离子空间电荷区的电荷密度比电子空间电荷区大得多,使得整个极间电压几乎全部集中在阴极附近的狭窄区域内。这是辉光放电的显著特征,而且在正常辉光放电时,两极间电压不随电流变化。

在阴极附近,二次电子发射产生的电子在较短距离内尚未得到足够的能使气体分子电离或激发的动能,所以紧接阴极的区域不发光。而在阴极辉区,电子已获得足够的能量碰撞气体分子,使之电离或激发发光。其余暗区和辉区的形成也主要取决于电子到达该区的动能以及气体的压强(电子与气体分子的非弹性碰撞会失去动能)。

4.四极质谱仪中四极电场特性如何?其中离子作何运动?

四极质谱仪中是一个周期性的电场,在$x$方向粒子受到的电场力为
\beq
F_X = -e\frac{\Phi_0 x}{r_0^2}
\eeq
其中$\Phi = -(U-V\cos\Omega t)$是一个周期震荡的场。粒子在$x$方向的运动满足马修方程
\beq
\frac{d^2x}{dt^2}+\left(\frac{2eU}{mr_0^2}+\frac{2eV\cos\Omega t}{mr_0^2}\right)x=0
\eeq
根据上面对马修方程的研究,在$z$轴方向气体分子不受到作用力,故离子在$z$轴方向作匀速运动,在$x,y$平面上,如果离子的荷质比满足一定的条件~\ref{eq:condition},那么离子在$z$轴附近做运动,但如果离子的荷质比不满足一定条件,那么在$x,y$方向的运动就是发散的,从而撞在仪器边缘上,从而被筛选掉。
\bibliographystyle{siam}
\bibliography{cites}
\end{document}
